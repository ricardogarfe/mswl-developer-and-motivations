\documentclass[11pt]{scrartcl}
\usepackage[utf8]{inputenc}
\usepackage[spanish]{babel}

\title{\textbf{FLOSS Communities}}
\subtitle{Roles and Organizational Policies}
\author{Ricardo García Fernández}
\date{\today}

\begin{document}

\maketitle

\section{Introduction to Spring}

Spring es un Framework Open Source (\emph{Apache License Version 2.0}\footnote{http://www.springsource.org/spring-framework}) para facilitar el desarrollo de proyectos utilizando Java aplicando las mejores pr\'acticas de dise\~no. Creado a principios de la primera d\'ecada de los a\~nos 2000 y apareci\'o como publicado en Sourceforge en 2003 de la mano de Rod Johnson\footnote{http://www.vmware.com/es/company/leadership/rod-johnson.html}, su creador y que este mismo año ha abandonado el proyecto para embarcarse en una nueva acometida\footnote{http://blog.springsource.org/2012/07/03/oh-the-places-youll-go/}.

Rod Johnson diseñó un framework para facilitar el desarrollo de proyectos utilizando Java aplicando las mejores prácticas de diseño. Fue creado durante el periodo de recopilación de buenas prácticas que recoge el libro \emph{Expert One-on-one J2EE Design And Development (Programmer to programmer)} para trasladarlas mediante la simplicidad al desarrollo de aplicaciones.

Formó un pequeño equipo para extender las funcionalidades del framework publicando la primera versión, que no la \emph{version 1.0}, en Sourceforge en febrero de 2003. El lanzamiento de la versión 1.0 en Marzo de 2004 fue muy bien aceptada por la comunidad debido a una característica muy importante en este tipo de proyectos, la documentación. La cantidad y calidad de documentación que acompañaba al Framewrok ayudó a incrementar el uso en la comunidad, usuarios particulares y proyectos empresariales.

Esta característica se sigue manteniendo y evolucionando día a día, mediante, tutoriales, foros, tickets, vídeos, charlas, certificados, en resumen, una muy buena comunicación con los usuarios del producto por lo que mantiene una gran comunidad asociada y potente.

El pequeño grupo de desarrolladores fue creciendo hasta desembocar en lo que hoy conocemos como SprinSource la organización que está detrás del Framework Spring. Hace unos años en 2009 fue adquirida por VMWare\footnote{http://www.vmware.com/company/news/releases/springsource.html}, por lo que hereda de la empresa la jerarquía pero no el control interno, es decir sigue creciendo como Framework de desarrollo libre orientado a la comunidad.

\section{Roles y comunidad}

Empecemos por el principio ¿ Que es un Rol ? Según la RAE: \emph{funci\'on que alguien o algo cumple.}
Por lo tanto un Rol es la función que desempeña una persona, en este caso dentro de un proyecto.
Vamos a ver que tipos de Roles se pueden enmarcar dentro de SpringSource.

Podemos ver una lista genérica a través de la información que nos facilita Spring:

\begin{itemize}
    \item Usuario - El usuario de Spring que utiliza la herramienta para el desarrollo abasteciéndose de la información de todos los tipos de manuales.
    \item Usuario activo - El usuario de Spring que se involucra en la comunidad registrándose en el foro, JIRA, StackOverFlow, etc, para buscar/preguntar una duda que le haya surgido.
    \item Usuario pro-activo - El usuario que además de preguntar, contesta a las dudas expuestas a través de los diferentes canales incluso subiendo parches que no aplicará el mismo directamente. Este apartado merece una mención especial ya que es el más cuidado y más importante dentro de la comunidad.
    \item Usuario JIRA - El usuario que propone mejoras a la herramienta mediante JIRA abriendo debates mediante tickets. éstos tickets y cualquier otro son votados por la comunidad de JIRA para la implementación de los mismos o resolución de problemas entre la comunidad para poder establecer un orden de prioridad.
    \item Usuario Committer - El usuario que ayuda gestionando y resolviendo tickets que le asigna el grupo de desarrolladores encargados del proyecto. Aporta el código fuente directamente habiendo aceptando el contrato para la subida de código al repositorio de Spring siguiendo las directrices establecidas para la documentación y el formato.
\end{itemize}

Se aprecia una nube de Roles característicos para todo aquel desarrollador de software interesado en participar en la comunidad. Este esquema de roles comprende desde usuario que utiliza Spring (o alguno de sus productos satélite) hasta el ponente en el congreso anual de Spring.

No hay diferencia entre un usuario que esté trabajando en Spring y un usuario que esté indentificado con estos roles (igual el sueldo si) ya que están dentro de una comunidad para sacar adelante el proyecto por lo que el intercambio de roles de un usuario depende de la actividad en la que éste se implique en la comunidad.

\section{Primer vistazo}

Al acceder a la página del proyecto de Spring podemos ver un mensaje que invita a entrar y colaborar mediante dos vistosos botones: \emph{Get Started}\footnote{http://www.springsource.org/get-started} y \emph{Get Involved}\footnote{http://www.springsource.org/get-involved} por lo que por otra parte existen dos tipos de camino, el nuevo usuario y el usuario avanzado o habitual para interactuar con la comunidad.

Para comprender la estructura de la comunidad asociada al proyecto empezaremos analizando el camino del nuevo usuario.

\section{Get Started}

En este apartado de iniciación podemos encontrar todo tipo de información referente al aprendizaje del proyecto Spring dividido en estos seis grupos:

\begin{itemize}
    \item Start a Tutorial - Tutoriales de uso.
    \item Grab a Code Sample - Ejemplos funcionales.
    \item Ask a Question (Forums) - Foro, activo e importante.
    \item Take a Class (Training) - "Universidad" de Spring para aprender a codificar.
    \item Read the Documentation - Importantísimo apartado, no sólo leer si no, saber utilizar la documentación.
    \item Video Instruction - Vídeos en los que muestra las herramientas y su uso en funcionamiento.
\end{itemize}

Como hemos remarcado anteriormente, Spring destacó y destaca por la calidad de su documentación y apoyo para la adopción del Framework por parte de los desarrolladores.
Este proyecto nos invita a participar facilitando su comprensión a un desarrollador Java desde distintos escalones de conocimiento.

Intenta abarcar diferentes métodos de aprendizaje, lecturas, ejemplos, vídeos, preguntas, clases virtuales para que de esta manera no sólo exista una forma de comprensión y los desarrolladores puedan elegir el camino que más se adapte a ellos.

Se puede definir como una comunidad de aprendizaje del Framework Spring dentro de la misma comunidad Spring. Una subcomunidad en donde los usuario aprenden y pueden convertirse en profesores (utilizando la nomenclatura de la enseñanza) alrededor de un foro en el que orbitan los demás servicios comunicándose entre ellos.

\section{Get Involved}

Aquí es donde empieza el camino del usuario habitual o avanzado.

La presentación de esta página nos muestra un entusiasta comentario de bienvenida a la comunidad no importa el nivel de conocimiento que tengas de Spring ya que este es un grupo dedicado a aprender y a difundir los conocimientos.

Está más marcada la estructura escalonada por pasos con lo que se refriere a roles para un usuario inicial en esta etapa:

\begin{itemize}
    \item Join the conversarion - Anima a entrar en el ecosistema Spring para recibir información del día a día.
    \item Help other users (And get help when you need it too) - Uso de los foros para ayudar o recibir ayuda a los demás y también mediante el uso de la red social StackOverFlow\footnote{http://stackoverflow.com/} con los tags \emph{spring} y \emph{spring-mvc}.
    \item Report issues - Informa de errores o mejoras a través del JIRA\footnote{http://jira.springsource.org/} de Spring.
    \item Track the latest features and test them out - Uso activo del JIRA para poder probar las nuevas características o posibles errores ayudando a la comunidad a resolverlos.
    \item Contribute code - Nos emplaza al repositorio de código de Spring en GitHub\footnote{http://github.com/SpringSource} para utilizar la última versión del código fuente publicado para que a través de los tickets definidos en Github y su relación con el JIRA puedas aportar una solución o mejora a través de tu código fuente y si éste soluciona o satisface los requisitos, entrarás a formar parte del grupo de usuarios que aportan código a Spring Framework mediante la firma de un contrato de contribuyente \footnote{https://support.springsource.com/spring\_committer\_signup} al proyecto.
    \item Attend (or give) a talk at a local user group - Anima al desarrollador a ir a charlas sobre Spring o a ser el ponente de la misma charla en tus grupos cercanos de desarrolladores destacando que Spring aprecia mucho esta tarea, la ayudar a promover el buen uso de su Framework ofreciendo directamente su ayuda.
    \item Attend a spring-related conference - Como último punto nos da información sobre la conferencia anual de Spring donde se informa y se debate sobre las mejoras y/o novedades de Spring uniéndote a ella para estar más en contacto con la comunidad.
\end{itemize}

\section{Organigrama de la comunidad}

Aquí podemos observar el recorrido de interacción de un usuario a través de los diferentes roles o estados de la comunidad entre los que puede ir saltando.\\
Desde el usuario básico de Spring que obtiene conocimiento a través de la documentación, interactúa con el foro, crea una propuesta en JIRA (la cual ha de promover para publicitar en la comunidad) y encontrar una resolución.\\
Como hemos visto antes los Roles encargados de resolver las tareas relacionadas son definidos a través de la comunidad.\\
Por lo que cada rol tiene especificaciones definidas para su contexto que Spring se esfuerza en facilitar y promocionar por ello es una comunidad muy activa.

\section{Conferencias}

\begin{enumerate}
\item Spring2gx: Conferencia Oficial de Spring\footnote{http://www.springone2gx.com/conference/washington/2012/10/home}. Se celebra anualmente, este año el lugar elegido es Washington a final de año. La entrada a este evento es costosa y el objetivo es abordar las novedades que esperan a la comunidad desde el nivel más alto hasta el inicial haciendo un resumen anual. Existen talleres, charlas y propuestas dentro de la misma conferencia.
\item SpringIO: Conferencia Internacional que se celebra en Madrid a trav\'es de JavaHispano\footnote{http://www.javahispano.org/portada/2012/6/12/mas-videos-del-spring-io-2012.html}. Esta conferencia que se celebra en España tiene como objetivo acercar al desarrollador a las nuevas innovaciones del Framework mediante charlas y talleres específicos en un foro interactivo donde confluye gente con los mismo intereses, acercando a un mismo nivel, desde los desarrolladores internacionales hasta el usuario básico.
\end{enumerate}



\begin{thebibliography}{9}

\bibitem{spring-inside}
  SpringSource Community,\\
  http://www.springsource.org/

\end{thebibliography}
\end{document}
