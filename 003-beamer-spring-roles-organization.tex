\documentclass[xcolor=dvipsnames]{beamer}
\usecolortheme[named=Green]{structure}
\usetheme{Warsaw}
\usepackage{beamerthemesplit}
\usepackage{latexsym}
\usepackage{eurosym}
\usepackage{ae,aecompl}
\usepackage{graphicx}
\usepackage{amsfonts}
\usepackage{tikz}

\usetheme{Darmstadt}

\title{Spring Source Roles Organization}
\titlegraphic{\includegraphics{logo_springsource_community.png}}
\author{Ricardo Garc\'ia Fern\'andez,\\
        Developers and Motivations,\\
        Universidad Rey Juan Carlos I.}
\date{\today}

\begin{document}


\frame{\titlepage
\begin{flushright}
{\tiny
(cc) 2012 Ricardo Garc\'ia Fern\'andez\\
    Este obra está bajo una licencia de Creative Commons Reconocimiento 3.0 Unported.
    To view a copy of full license, see http://creativecommons.org/licenses/by/3.0/
}
\end{flushright}
}

%% \usebackgroundtemplate{
%%   \parbox[c][\paperheight][c]{\paperwidth}{\centering\includegraphics[width=5.5in]{logo_springsource_community.png}}
%% }

\section[\'Indice]{}
\begin{frame}[allowframebreaks]
\tableofcontents
\end{frame}

\section{Spring Source Organization}
\begin{frame}[allowframebreaks]
\frametitle{Histioria de Spring}
Spring es un Framework Open Source (\emph{Apache License Version 2.0}\footnote{http://www.springsource.org/spring-framework}) para facilitar el desarrollo de proyectos utilizando Java aplicando las mejores pr\'acticas de dise\~no. Creado a principios de la primera d\'ecada de los a\~nos 2000 y apareci\'o como publicado en Sourceforge en 2003 de la mano de Rod Johnson\footnote{http://www.vmware.com/es/company/leadership/rod-johnson.html}, su creador.
\end{frame}

\section{Roles en la comunidad.}
\begin{frame}[allowframebreaks]
\frametitle{Definici\'on de Rol}
Definici\'on de Rol obtenida de la RAE: \emph{funci\'on que alguien o algo cumple.}
\end{frame}

\begin{frame}[allowframebreaks]
\frametitle{Lista de los principales Roles}
Con respecto a los roles, podemos extraer una lista gen\'erica a trav\'es de la informaci\'on que nos facilita Spring en su descripci\'on de 'quehaceres' dentro de la comunidad:
\begin{itemize}
    \item \emph{Usuario} - El usuario de Spring como herramienta.
    \item \emph{Usuario activo} - Usuario que busca respuestas a preguntas concretas
    \item \emph{Usuario pro-activo} - Usuario que aporta respuestas e intercambia conocimientos.
    \item \emph{Usuario JIRA} - Usuario que comprende y propone mejoras de la herramienta.
    \item \emph{Usuario Committer} - Usuario resolutivo de problemas y generador de mejoras.
\end{itemize}
\end{frame}

\section{Interacci\'on en la comunidad}
\begin{frame}[allowframebreaks]
\frametitle{Pol\'tica de interacci\'on en la comunidad}
\begin{itemize}
    \item \emph{Get Started}\footnote{http://www.springsource.org/get-started}
    \item \emph{Get Involved}\footnote{http://www.springsource.org/get-involved}
\end{itemize}
\end{frame}

\begin{frame}[allowframebreaks]
\frametitle{Get Started}
\begin{itemize}
    \item \emph{Start a Tutorial} - Tutoriales de uso.
    \item \emph{Grab a Code Sample} - Ejemplos funcionales.
    \item \emph{Ask a Question (Forums)} - Foro, activo e importante.
    \item \emph{Take a Class (Training)} - "Universidad" de Spring para aprender a codificar.
    \item \emph{Read the Documentation} - Important\'isimo apartado, no s\'olo leer si no, saber utilizar la documentaci\'on.
    \item \emph{Video Instruction} - V\'ideos en los que muestra las herramientas y su uso en funcionamiento.
\end{itemize}
\end{frame}

\begin{frame}[allowframebreaks]
\frametitle{Get Involved}
\begin{itemize}
    \item \emph{Join the conversarion} - Informaci\'on del d\'ia a d\'ia.
    \item \emph{Help other users (And get help when you need it too)} - Uso de los foros y otros canales.
    \item \emph{Report issues} - Informa de errores o mejoras.
    \item \emph{Track the latest features and test them out} - Uso activo del JIRA para poder probar las nuevas caracter\'isticas o posibles errores.
    \item \emph{Contribute code} - Formar parte del grupo de usuarios que aportan c\'odigo a Spring Framework.
    \item \emph{Attend (or give) a talk at a local user group} - Anima al desarrollador a ir a charlas sobre Spring o a ser el ponente de la misma charla en tus grupos cercanos de desarrolladores.
\end{itemize}
\end{frame}

\section{Organigrama}
\begin{frame}[allowframebreaks]
\frametitle{Organigrama de la comunidad}
Dise\~no de la comunicaci\'on dentro de la comunidad en el que intervienen todos los roles.
\begin{itemize}
    \item Usuario - Documentaci\'on - Foro - JIRA - Votaci\'on - Resoluci\'on.
\end{itemize}
\emph{Todo el proceso viene apoyado a trav\'es del seguimiento de la comunidad mediante su participaci\'on.}
\end{frame}

\section{Conferencias}
\begin{frame}[allowframebreaks]
\frametitle{Conferencias sobre Spring}
\begin{enumerate}
\item Spring2gx: Conferencia Oficial de Spring\footnote{http://www.springone2gx.com/conference/washington/2012/10/home}.
\item SpringIO: Conferencia Internacional que se celebra en Madrid a trav\'es de JavaHispano\footnote{http://www.javahispano.org/portada/2012/6/12/mas-videos-del-spring-io-2012.html}
\end{enumerate}
\end{frame}

\end{document}
