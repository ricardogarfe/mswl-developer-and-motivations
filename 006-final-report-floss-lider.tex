\documentclass[11pt]{scrartcl}
\usepackage[utf8]{inputenc}
\usepackage[spanish]{babel}
\usepackage{graphicx}
\usepackage{float}

\title{\textbf{Guido Van Rossum - My dream was to build my own calculator}}
\subtitle{Developers and Motivations - MSWL Universidad Rey Juan Carlos I}
\author{Ricardo García Fernández}
\date{\today}

\begin{document}

\maketitle

\newpage

\tableofcontents

\newpage

\section{Guido van Rossum and Python in a nutshell}

Guido van Rossum es el creador de Python, uno de los lenguajes de programación más utilizados alrededor del mundo\footnote{http://www.tiobe.com/index.php/content/paperinfo/tpci/index.html}. Es un lenguaje de programación bajo una Licencia Libre \emph{Python Software Foundation License -PSFL-}\footnote{http://docs.python.org/2/license.html}. Está respaldado por una comunidad grande que evoluciona según las necesidades de los usuarios a través de la organización \emph{Python Software Foundation} de la que Guido es su actual presidente, el cual actúa como \emph{Benevolent Dictator for Life -BDFL-}\footnote{BDFL no se debe confundir con el término usado por Eric Raymon en su ensayo sobre los Líderes de los proyecto de Software Libre el cual los reconoce como dictadores benevolentes en 1997. Este sobrenombre se le acuñó durante un workshop en 1995 en el resumen del mismo:http://www.artima.com/weblogs/viewpost.jsp?thread=235725} y apoyando la comunidad a través de las propuestas y recomendaciones definidas a través de los \emph{Python Enhancement Proposal - PEPs}\footnote{http://www.python.org/dev/peps/}.

El nombre del lenguaje Python proviene del nombre del grupo de cómicos ingleses Monty Python\footnote{http://www.imdb.com/title/tt0063929/} a diferencia de lo que la gente pueda pensar, no tiene nada que ver con serpientes.

\section{Biography}

Biography and relevant details to contextualize their contributions in FLOSS.

Guido Van Rossum nació el 31 de Enero de 1956 en los Países Bajos. Estudió la carrera de matemáticas y ciencias de la computació en la Universidad de Ámsterdam\footnote{http://www.uva.nl/en} \emph{-Curiosamente es la misma Universidad en la que Linus Torvalds desarrolló el primer esbozo del conocido sistema operativo-}.
No tuvo acceso a un ordenador hasta que no entró en la Universidad a los 18 años, para el era algo alucinante pero que tenía que compartir con los demás alumnos.

Empezó a trabajar en la Universida con las tarjetas perforadas, sus primeros pasos en la comunicación con una máquina. Más adelante se rodeó de un grupo de programadores que se reunía en el sótano de la facultad en donde se encontraban las lectoras de tarjetas, las impresoras y ellos. Hablando, discutiendo, al fin y al cabo aprendiendo unos de otros alrededor de un tema que les ilusionaba, la programación.

Guido creó su primer "Hello World!" a través del lenguaje Algol-60\footnote{http://www.masswerk.at/algol60/report.htm} algo tosco para el pero que le ayudó a entrar en el mundo de la programación. El siguiente lenguaje de programación en conocer fue Pascal a través de la recomendación de un profesor, fue directo a conseguir nuevos conocimientos del único libro que había en la Universidad\footnote{http://www.anvari.org/fortune/Miscellaneous\_Collections/228172\_silver-book-n-jensen-and-wirths-infamous-pascal-user-manual-and-repo.html} y se quedó con lo más significativo.

Hay una frase escrita por Guido que describe el proceso de aprendizaje de un lenguaje como Pascal:

\begin{quote}
    Pascal really had only one new feature compared to Algol-60, pointers. These baffled me for the longest time. Eventually I learned assembly programming, which explained the memory model of a computer for the first time. 
    I realized that a pointer was just an address. Then I finally understood them.
\end{quote}

En la que vemos que comprende claramente los fundamentos del lenguaje. Tiene una habilidad para con los lenguajes y por supuesto interés en aprenderlos. Podríamos definir a Guido como un políglota ya que poco a poco se sumaban más lenguajes de programación en su haber; \emph{Fortran, Lisp, Basic, Cobol}. Los lenguajes de programación le atraían más que la programación en si, la comunicación y la forma de ejercerla con la máquina.

El conocimiento de una gran variedad de lenguajes de programación existentes en los 80 le llevó a trabajar para el data center de la Universidad manteniendo el sistema operativo. Aquí tuvo el primer contacto con un sistema Unix donde aprendió C y Shell Scripting. Una de las cosas a destacar de esta posición es que le permitió el acceso directo a una computadora. Programaba porque le gustaba, por placer no por necesidad de crear nada en cuestión, es decir no tenía un \emph{leitmotive}\footnote{http://es.wikipedia.org/wiki/Leitmotiv}.

Hasta que la unión entre sus conocimientos matemáticos junto al interés por los lenguajes de programación, propicionados en el lugar donde se encontraba estudiando y trabajando, tuvo como resultado su inclusión en el proyecto \emph{ABC group}\footnote{http://homepages.cwi.nl/~steven/abc/} en 1982. ABC es un lenguaje interactivo de programación. Empezó su trabajo en el \emph{"Centre for Mathematics and Computer Science - CWI"} de la Universidad de Amsterdad en donde se puede clasificar como uno de los puntos de inflexión en la historia de Guido.

Su experiencia obtenida en el desarrollo de ABC durante 4 años le sirvió para embarcarse en el siguiente proyecto Amoeba\footnote{http://www.cs.vu.nl/pub/amoeba/}, un sistema operativo distribuido. Un proyecto de colaboracioń entre el CWI y la Universidad Vrije de Amsterdam (Universidad Libre)\footnote{http://www.vu.nl/en/index.asp}.

\begin{quotation}
    Python is a direct product of my experience at CWI. As I explain later, ABC gave me the key inspiration for Python, Amoeba the immediate motivation, and the multimedia group fostered its Michael McLaygrowth.
\end{quotation}

Aquí es donde Guido encuentra su motivación, según sus propias palabras:
\begin{quotation}
My original motivation for creating Python was the perceived need for a higher level language in the Amoeba project. I realized that the development of system administration utilities in C was taking too long
\end{quotation}

Desestimó la portabilidad de cualquier otro lenguaje, incluso de Perl 3 debido a que está fuertemente ligado a Unix y no le gustaba su sintaxis, dato importante que se verá reflejado en Python, la importancia del a sintaxis que está fuertemente influenciado por los lenguajes Algol 60, Pascal, Algol 68.
Comienza el desarrollo de Python a finales de 1989. El desarrollo de Python se basa en ABC corrigiendo los errores que contenía bajo su perspectiva. Se publicó internamente la primera versión funcional en el primer tercio del año 1990 en la cual recuerda con cariño sus primeras funcionalidades de las cuales aún perduran como \emph{pgen}.

\section{Project evolution}

Se introdujo el uso del sistema de control de versiones \emph{CVS} que fue creado por uno de sus compañeros del \emph{proyecto ABC}, Dick Grune\footnote{http://www.dickgrune.com/} y Guido creó un FAQ que distribuía a través de distintas listas de correo.

En 1993 se creó la propia lista de correo que todavía sigue activa y bajo un software hecho en Python. La eplosión de la comunicación de los usuarios mediante la lista de correo centralizada significó la expansión en si misma de Python. La lista de correo fue el siguiente paso de la evolución a patir de un simple mensaje en el que el asunto es impactante y encierra mucho más significado:

\begin{quote}
If Guido was hit by a bus?
\end{quote}

Este mensaje proviene de \emph{Michael McLay}\footnote{http://www.python.org/search/hypermail/python-1994q2/1040.html} y traslada esa cuestión a Guido, \emph{¿ que pasaría con Python si atropella a Guido un autobús ?}. La evolución del sistema estaba centralizada a través de Guido por lo que si el desaparecía, el proyecto caería en una recesión no deseada.
El cuerpo del mensaje también explica un parte importante en la evolución de Python, la estandarización del lenguaje, en otras palabras la profesionalización. Debido que si se ha de extender el uso de Python ha de cumplir una serie de estándares para dotarlo de una fiabilidad para que las empresas, organizaciones, etc, lo adopten como su tecnología transmitiéndoles seguridad, robustez y madurez.

Guido fue invitado a trabajar en el \emph{US National Institute for Standards and Technology NIST} a través de M.McLay para seguir evolucionando Python y promoverlo mediante workshops dentro de los EEUU a través del NIST.

El primer workshop se llevó a cabo en Noviembre de 1994 junto \emph{Ken Manheimer}\footnote{http://myriadicity.net/}, un desarrollador del NIST, a la que asistieron 20 personas, de las cuales la mitad sigue participando activamente en la comunidad Python e incluso unos pocos se han convertido en líderes FLOSS de proyecto propios(Jim Fulton de Zope y Barry Warsaw de GNU mailman). A través del soporte y la promoción desarrollada en el NIST Guido acabó publicitando workshops delante de más de 400 asistentes lo que hacía que creciera mediante la difusión el conocimiento de Python.

Annotated description of the evolution of the FLOSS project they lead, or have led in the past.
Conversion to a FLOSS community project.

Primeros usuarios.
Listas de correo.

Expansión a NCIS.
Creación de Comunidad.

http://python-history.blogspot.com.es/2009/01/brief-timeline-of-python.html

\subsection{Python Software Foundation}

Python Software Foundation\footnote{http://www.python.org/psf/} es la fundación de Python que fue constituida en 2001\footnote{http://www.python.org/psf/press-release/pr200103/}. Se anunció durante el noveno año de la conferencia anual de Pythom a través de Guido siguiendo los pasos del modelo creado por la Apache Software Foundation.
Las funciones destacadas están relacionadas con la educación, el mantenimiento de la web y el fomento del uso de Python a través de sus grupos. Y como más significativo, Guido le otorga la responsabilidad, es decir los derechos de propiedad intelectual haciendo así el traspaso de poder y liberando Python totalmente a la comunidad.

Y no, no fue Guido el primer presidente, en este caso fue \emph{Dick Hardt} por parte de la una de las primeras empresas en esponsorizar la PSF, ActiveState\footnote{http://www.activestate.com/}.

\section{First contact}

Founder? Arrived lately?

My original motivation for creating Python was the perceived need for a higher level language in the Amoeba project. I realized that the development of system administration utilities in C was taking too long. 

\section{Contributions}

If possible, try to quantify their contributions (number of commits, bug fixes or other relevant activities to boost the project).

oholoh - http://www.ohloh.net/p/python and Guido profile http://www.ohloh.net/p/python/contributors/113816964319

Para hacer un análisis a groso modo del estado de la comunidad vamos a utilizar el repositorio de código fuente\footnote{http://hg.python.org/cpython}, el bugtracker\footnote{http://bugs.python.org/}, los archivos de las listas de correo\footnote{http://mail.python.org/pipermail/python-list/} y las PEP\footnote{http://www.python.org/dev/peps/}.


\subsection{SCM repository}

Mediante el uso de la herramienta online Ohloh podemos ver fácilmente un histórico del repositorio distribuido Mercurial de Python \emph{http://www.ohloh.net/p/python}.

Podemos observar desde los inicios únicamente existía un contribuidor al proyecto, Guido. A partir de Julio 1992 se aprecian dos nuevos contribuyentes más que van alternando su trabajo él durante un año llegando a estabilizar esta cifra durante 6 meses, desde Septiembre de 1993 a Ferbrero de 1994. En junio de 1994 se produce el famoso mensaje de Michael McLay y esto parece que repercute en la vida del proyecto haciendo crecer el número de desarrolladores base a dos a principios de 1995. 

Cuando se muda a EEUU para trabajar durante 5 años en el NIST evolucionando Python con un nuevo grupo de desarrolladores lo contribuyentes básicos crecen linealmente, desde el mínimo de 2 hasta un máximo de 8 al empezar el año 2000 estabilizando un mínimo de usuarios mayor que 10 en ese mismo año llegando a existir 21 diferentes contribuyentes. El trabajo hecho a través del NIST, desarrollo, workshopsks y la gestión de la comunidad mediante listas de correo apoyados por la distribución del código fuente a través de CVS consolidaron la autosuficiencia de la comunidad.

Como siguiente paso significativo dentro del mismo año 2000 en Julio se crearon los PEPs\footnote{http://www.python.org/dev/peps/}


\subsection{Mail list}

que el proyecto a partir de la creción de la PSF duplicó el número de contribuyentes al mismo de 

Y por supuesto existe un perfil de Guido asociado al proyecto Python\footnote{http://www.ohloh.net/p/python/contributors/113816964319}. Del que podemos destacar que el último año únicamente ha hecho dos aportaciones al núcleo pero sigue siendo el usuario que más commits ha aportado durante toda la historia.

\subsection{Workshops}



\subsection{Issue tracker}

First issue: http://bugs.python.org/issue1034

\section{Is still Leading ?}

Is this person still leading the project? If not, why did he/she left?

Trabaja actualmente (2012) en Google dedicando el 50\% de su tiempo al desarrollo de Python desde su posición de \emph{BDFL}. Es el actual presidente de la \emph{PSF} desde 2002.

Es el director de la PSF. En el que su trabajo se ve reflejado más a través del rol de \emph{BDFL}, mediante charlas, ponencias, talleres que en el desarrollo del propio núcle de Python como se aprecia en el análisis del repositorio en donde únicamente aparecen 2 commits en el último año a su nombre\footnote{http://www.ohloh.net/p/python/contributors/113816964319} aunque eso sí, el primer commit es de 1990 y aporta un total de 10931 commits al proyecto siendo el que mayor número ha aportado\footnote{http://www.ohloh.net/p/python/contributors?query=\&sort=commits}.

Es una persona orgullosa de su trabajo, de que signifique algo y de que no sólo se quede ahí. Guido desde sus inicios con Python repite una y otra vez que la programación debería estar en el día a día de cada uno. Su intención de llevar a los institutos la asignatura de programación.

\section{Recognized rol}

Was he/she essential for the success of the project or the surrounding community? Why?

Guido consolidó el proyecto gradualmente junto a la comunidad. La primera distribución del proyecto se licenció mediante una Licencia de Software Libre (MIT) y la base de usuarios como hemos visto eran sus propios compañeros de despacho en Ámsterdam. Los usuariose se fueron expandiendo más allá de su despacho al empezar a utilizar la distribución publicada, la comunicación funcionaba ahora a través de la lista de correo.

La máxima a la que aspira Guido es conseguir que Python sea conocido y utilizado no sólo por los desarrolladores de software si no que también sea un lenguaje utilizado por la gente no técnia. Guido mantiene la idea en su mente que todo el mundo debería de aprender a programar durante sus estudios de una manera relativamente fácil. Es un medio para socializar la progrmación y así demostrar lo que aporta siendo una herramienta útil.

Computer Programming for Everybody, in which he further defined his goals for Python:
\begin{itemize}
    \item an easy and intuitive language just as powerful as major competitors
    \item open source, so anyone can contribute to its development
    \item code that is as understandable as plain English
    \item suitability for everyday tasks, allowing for short development times
\end{itemize}

\section{Curiosidades}

\begin{quote}Python's indentation was invented by the wife of Robert Dewar\end{quote}

If Guido was hit by a bus? www.python.org/search/hypermail/python-1994q2/1040.html

Se definió una filosofía del desarrollador en Python la cual se conoce como \emph{The Zen of Python} que fue implementada por Tim Peters un desarrollador muy importante dentro del núcleo de Python y publicada como un PEP0020\footnote{http://www.python.org/dev/peps/pep-0020/}. En ella se describen las buenas prácticas del desarrollo mediante Python en un lenguaje coloquial se podría decir, inlcuso una especie mandamientos trascritos a una narrativa cercana.

Python is the \#3 most popular language on GitHub https://github.com/languages/Python

\section{Contact information}

\begin{thebibliography}{9}

    \bibitem{guidopersonalhomepage}
      Personal home page,\\
      Guido van Rossum,\\
      http://www.python.org/~guido/
    
    \bibitem{pythonhistory}
        The History of Python,\\
        Guido van Rossum \& Greg Stein,\\
        http://python-history.blogspot.com

    \bibitem{pytalent}
        Meet the Mighty Python,\\
        http://pytalent.zandstrasystems.com/meet\_py1.html
\end{thebibliography}
\end{document}
