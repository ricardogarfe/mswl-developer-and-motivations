\documentclass[11pt]{scrartcl}
\usepackage[utf8]{inputenc}
\usepackage[spanish]{babel}
\usepackage{graphicx}
\usepackage{float}

\title{\textbf{Guido Van Rossum - FLOSS Leader}}
\subtitle{Developers and Motivations - MSWL Universidad Rey Juan Carlos I}
\author{Ricardo García Fernández}
\date{\today}

\begin{document}

\maketitle

\newpage

\tableofcontents

\newpage

\section{Enunciate}

Specific written report (5-10 pages) about a prominent leader in FLOSS. An initial list of candidate profiles to choose from will be published in short on the course website. It is important to detail all the references, and to heavily root the report on data and/or specific works publicly available.

Max. length of the document (including graphs, tables, etc., but excluding the front page) is 10 pages.

Some aspects that must be considered include:
\begin{itemize}
    \item Biography and relevant details to contextualize their contributions in FLOSS.
    \item Annotated description of the evolution of the FLOSS project they lead, or have led in the past.
    \item Founder? Arrived lately?
    \item If possible, try to quantify their contributions (number of commits, bug fixes or other relevant activities to boost the project).
    \item Is this person still leading the project? If not, why did he/she left?
    \item Was he/she essential for the success of the project or the surrounding community? Why?
\end{itemize}

It is possible that they have lead more than one project. Then, students can either focus on just one of these projects or enlighten the transitions between projects and their reasons. A thorough description of the main leadership traits exhibited by the leader is essential.

\section{Guido van Rossum as a FLOSS Lider}

\section{Biography}

Biography and relevant details to contextualize their contributions in FLOSS.

Untill he arrives at % TODO:

% @Vini: I didn't create the Zen of Python. Tim Peters wrote it. Barry Warsaw documented when:
% http://www.wefearchange.org/2010/06/import-this-and-zen-of-python.html

\section{Project evolution}

Annotated description of the evolution of the FLOSS project they lead, or have led in the past.

\subsection{Python Software Foundation}

http://www.python.org/psf/press-release/pr200103/

\section{First contact}

Founder? Arrived lately?

My original motivation for creating Python was the perceived need for a higher level language in the Amoeba project. I realized that the development of system administration utilities in C was taking too long. 

\section{Contributions}

If possible, try to quantify their contributions (number of commits, bug fixes or other relevant activities to boost the project).

\section{Is still Leading ?}

Is this person still leading the project? If not, why did he/she left?

Guido is president of Python Software Foundation since 2002. The first year the president was Dick Hardt.

\section{Recognized rol}

Was he/she essential for the success of the project or the surrounding community? Why?

\section{Contact information}

\section{Extras}

\begin{quote}Python's indentation was invented by the wife of Robert Dewar\end{quote}

If Guido was hit by a bus? www.python.org/search/hypermail/python-1994q2/1040.html

\begin{thebibliography}{9}

    \bibitem{omgubuntu}
      Interview in OMGUbuntu,\\
      Benjamin Kerensa,\\
      http://www.omgubuntu.co.uk/2012/08/omg-interviews-leann-ogasawara-canonical-kernel-team-manager-marathoner-and-mother
    \bibitem{canonical}
    Canonical profile,\\
    Leann Ogasawara,\\
    https://wiki.ubuntu.com/LeannOgasawara

\end{thebibliography}
\end{document}
