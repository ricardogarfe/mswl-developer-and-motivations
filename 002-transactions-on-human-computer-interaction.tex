\documentclass[11pt]{scrartcl}
\usepackage{atbegshi,picture}
\usepackage{lipsum}

\AtBeginShipout{\AtBeginShipoutUpperLeft{%
  \put(\dimexpr\paperwidth-1cm\relax,-1.5cm){\makebox[0pt][r]{Master on Free Software 2012/2013}}%
}}

\title{\textbf{Developers and motivations}}
\subtitle{Transactions on Human-Computer Interaction}

\author{Felipe Ortega\\
		Ricardo Grac\'ia Fern\'andez}
\date{\today}
\begin{document}

\maketitle

\section{Ideas}

	\begin{enumerate}
	
		\item A small fraction of users becomes leaders, who participate in governance by setting and upholding policies, repairing vandalized materials, or mentoring novices.
		\item reading, contributing, collaborating, and leading.
		\item usability and sociability factors.\\

		\begin{tabular}{ | p{6,5cm} | p{6,5cm} |}
			\hline
			Usability & Sociability \\ \hline
			Interesting and relevant content presented in attractive, well-organized layouts & Encouragement by friends, family, respected authorities, advertising \\ \hline
			Frequently updated content with highlighting to encourage return visits & Repeated visibility in online, print, television and other media \\ \hline
			Support for newcomers through tutorials, animated demos, FAQs, help, mentors, contacts & Understandable and clear norms or policies \\ \hline
			Clear navigation paths so that users have a sense of mastery and control & A sense of belonging based on recognition of familiar people and activities \\ \hline
			Universal usability to support novice/expert, small/large display, slow/fast network, multilingual, and users with disabilities & Charismatic leaders with visionary goals \\ \hline
			Interface design features to support reading, browsing, searching, and sharing & Safety and privacy \\ \hline
		\end{tabular}
					
		\item Usability and sociability factors that may influence contributing.\\

		\begin{tabular}{ | p{6,5cm} | p{6,5cm} |}
			\hline
			Usability & Sociability\\ \hline
			Low threshold interfaces for easily making small contributions, e.g., no login & Support for legitimate peripheral participation so that readers can gradually edge into contributing\\ \hline
			High ceiling interfaces that allow large and frequent contributions & A chance to build their reputation over time while performing satisfying tasks\\ \hline
			Visibility for users’ contributions and frequency of views; aggregated over time & Recognition for the highest quality and quantity of contributions\\ \hline
			Visibility of ratings and comments by community members & Recognition of a person’s specific expertise\\ \hline
			Tools to undo vandalism, limit malicious users, control pornography and libel & Policies and norms for appropriate contributions\\ \hline
		\end{tabular}

		\item Sometimes people shift quickly from contribution to collaboration and back again. For example, an ornithologist contributor to Wikipedia bird articles may be closely collaborating with a group of bird watchers in making sure that an entry about Greater Scaups on the Chesapeake Bay is correct. But she gets distracted by a friend’s email to read an entry about a café in London, whose address is listed as “Upper Road” in Islington, so she corrects it to “Upper Street.” In the first instance, she is involved in a collaboration in which she learns who has a deep knowledge about wildlife on the Bay. In the second instance, she merely contributes the correct address but does not interact with anyone.

		\item Usability and sociability factors that may influence collaborating.\\

		\begin{tabular}{ | p{6,5cm} | p{6,5cm} |}
			\hline
			Usability & Sociability\\ \hline
			Ways to locate relevant and competent individuals to form collaborations & An atmosphere of empathy and trust that promotes belonging to the community and willingness to work within groups to produce something larger\\ \hline
			Tools to collaborate: communicate within groups, schedule projects, assign tasks, share work products, request assistance & Altruism: a desire to support the community, desire to give back, willingness to reciprocate\\ \hline
			Visible recognition and rewards for collaborators, e.g., authorship, citations, links, acknowledgements & The desire to develop a reputation for themselves and their collaborators, their group or community; the need to develop and maintain one’s status within the group\\ \hline
			Ways to resolve differences (e.g., voting), mediate disputes, and deal with unhelpful collaborators & Respect for one’s status within the community\\ \hline
		\end{tabular}

		\item \textbf{Leader}: Promoting participation, mentoring novices, setting and upholding policies.
		
		\item Usability and sociability factors that may influence leadership.

		\begin{tabular}{ | p{6,5cm} | p{6,5cm} |}
			\hline
			Usability & Sociability\\ \hline
			Leaders are given higher visibility, and their efforts are highlighted, sometimes with historical narratives, special tributes, or rewards & Leadership is valued and given an honored position and expected to meet expectations\\ \hline
			Leaders are given special powers, e.g., to promote agendas, expend resources, or limit malicious users & Respect is offered for helping others and dealing with problems\\ \hline
			Mentorship efforts are visibly celebrated, e.g., with comments from mentees & Mentors are cultivated and encouraged\\ \hline
		\end{tabular}

	\end{enumerate}

\begin{thebibliography}{9}
	
	\bibitem{lamport94}
		Transactions on Human-Computer Interaction,\\
		Motivating Technology-Mediated Social Participation.\\
		\emph{Jennifer Preece}.\\
		\emph{Ben Shneiderman}.\\
		http://www.cs.umd.edu/~ben/papers/Jennifer2009Reader.pdf
\end{thebibliography}

\end{document}