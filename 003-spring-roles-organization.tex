\documentclass[11pt]{scrartcl}

\title{\textbf{FLOSS Communities}}
\subtitle{Roles and Organizational Policies}
\author{Ricardo Garc\'ia Fern\'andez}
\date{\today}

\begin{document}

\maketitle

\section{Exercise}

Analyze the \textbf{Spring} community to write a brief summary (1-2 pages) describing the main roles and organizational policies adopted in this community.

You must focus on the following points:

\begin{itemize}
    \item Different roles established in the community.
    \item Overall governance rules and guidelines.
    \item Promotion paths to access higher responsibilities in the community.
\end{itemize}

\section{Introduction to Spring}

Spring es un Framework Open Source bajo \emph{Apache License Version 2.0}\footnote{http://www.springsource.org/spring-framework} creado a principios de la primera d\'ecada de los a\~nos 2000. Apareci\'o como proyecto publicado en Sourceforge en 2003 de la mano de Rod Johnson, su creador y que este mismo a\~no ha abandonado el proyecto para embarcarse en una nueva acometida\footnote{http://blog.springsource.org/2012/07/03/oh-the-places-youll-go/}.

Rod Johnson dise\~n\'o un framework para facilitar el desarrollo de proyectos utilizando Java aplicando las mejores pr\'acticas de dise\~no para los proyectos.

El lanzamiento de la versi\'on 1.0 a la comunidad fue muy bien aceptada debido a una caracter\'istica muy importante en este tipo de proyectos, la documentaci\'on. La cantidad y calidad de documentaci\'on que acompa\~naba al Framewrok ayud\'o a incrementar el uso en la comunidad, usuarios particulares y proyectos empresariales.
Esta caracter\'istica se sigue manteniendo y evolucionando d\'ia a d\'ia, mediante, tutoriales, foros, tickets, v\'ideos, charlas, certificados, en resumen, una muy buena comunicaci\'on con los usuarios del producto por lo que mantiene una gran comunidad asociada y potente.

\section{Primer vistazo}

Al acceder a la p\'agina del proyecto de Spring podemos ver un mensaje que invita a entrar y colaborar mediante dos vistosos botones: \emph{Get Started}\footnote{http://www.springsource.org/get-started} y \emph{Get Involved}\footnote{http://www.springsource.org/get-involved} por lo que por otra parte existen dos tipos de camino, el nuevo usuario y el usuario avanzado o habitual para interactuar con la comunidad.

Para comprender la estructura de la comunidad asociada al proyecto empezaremos analizando el camino del nuevo usuario.

\section{Get Started}

En este apartado de iniciaci\'on podemos encontrar todo tipo de informaci\'on referente al aprendizaje del proyecto Spring dividido en estos seis grupos:

\begin{itemize}
    \item Start a Tutorial - Tutoriales de uso.
    \item Grab a Code Sample - Ejemplos funcionales.
    \item Ask a Question (Forums) - Foro, activo e importante.
    \item Take a Class (Training) - "Universidad" de Spring para aprender a codificar.
    \item Read the Documentation - Important\'isimo apartado, no s\'olo leer si no, saber utilizar la documentaci\'on.
    \item Video Instruction - V\'ideos en los que muestra las herramientas y su uso en funcionamiento.
\end{itemize}

Como hemos remarcado anteriormente, Spring destac\'o y destaca por la calidad de su documentaci\'on y apoyo para la adopci\'on del Framework por parte de los desarrolladores.
Este proyecto nos invita a participar facilitando su comprensi\'on a un desarrollador Java desde distintos escalones de conocimiento.

Intenta abarcar diferentes m\'etodos de aprendizaje, lecturas, ejemplos, vídeos, preguntas, clases virtuales para que de esta manera no s\'olo exista una forma de comprensi\'on y los desarrolladores puedan elegir el camino que m\'as se adapte a ellos.

Se puede definir como una comunidad de aprendizaje del Framework Spring dentro de la misma comunidad Spring. Una subcomunidad en donde los usuario aprenden y pueden convertirse en profesores (utilizando la nomenclatura de la ense\~nanza) alrededor de un foro en el que orbitan los dem\'as servicios comunic\'andose entre ellos.

\section{Get Involved}

Aqu\'i es donde empieza el camino del usuario habitual o avanzado.

La presentaci\'on de esta p\'agina nos muestra un entusiasta comentario de bienvenida a la comunidad no importa el nivel de conocimiento que tengas de Spring ya que este es un grupo dedicado a aprender y a difundir los conocimientos.

Est\'a m\'as marcada la estructura escalonada por pasos con lo que se refriere a roles para un usuario inicial en esta etapa:

\begin{itemize}
    \item Join the conversarion - Anima a entrar en el ecosistema Spring para recibir informaci\'on del d\'ia a d\'ia.
    \item Help other users (And get help when you need it too) - Uso de los foros para ayudar o recibir ayuda a los dem\'as y tambi\'en mediante el uso de la red social StackOverFlow\footnote{http://stackoverflow.com/} con los tags \emph{spring} y \emph{spring-mvc}.
    \item Report issues - Informa de errores o mejoras a trav\'es del JIRA\footnote{http://jira.springsource.org/} de Spring.
    \item Track the latest features and test them out - Uso activo del JIRA para poder probar las nuevas caracter\'isticas o posibles errores ayudando a la comunidad a resolverlos.
    \item Contribute code - Nos emplaza al repositorio de c\'odigo de Spring en GitHub\footnote{http://github.com/SpringSource} para utilizar la \'ultima versi\'on del c\'odigo fuente publicado para que a trav\'es de los tickets definidos en Github y su relaci\'on con el JIRA puedas aportar una soluci\'on o mejora a trav\'es de tu c\'odigo fuente y si \'este soluciona o satisface los requisitos, entrar\'as a formar parte del grupo de usuarios que aportan c\'odigo a Spring Framework mediante la firma de un contrato de contribuyente \footnote{https://support.springsource.com/spring\_committer\_signup} al proyecto.
    \item Attend (or give) a talk at a local user group - Anima al desarrollador a ir a charlas sobre Spring o a ser el ponente de la misma charla en tus grupos cercanos de desarrolladores destacando que Spring aprecia mucho esta tarea, la ayudar a promover el buen uso de su Framework ofreciendo directamente su ayuda.
    \item Attend a spring-related conference - Como \'ultimo punto nos da informaci\'on sobre la conferencia anual de Spring donde se informa y se debate sobre las mejoras y/o novedades de Spring uni\'endote a ella para estar m\'as en contacto con la comunidad.
\end{itemize}

\section{Roles y comunidad}

Despu\'es de la presentaci\'on de los Roles definidos en la web de Spring se aprecia una nube de Roles caracter\'isticos para todo aquel desarrollador de software interesado en participar en la comunidad. Este esquema de roles comprende desde usuario que utiliza Spring (o alguno de sus productos sat\'elite) hasta el ponente en el congreso anual de Spring.

Con respecto a los roles, podemos extraer una lista gen\'erica a trav\'es de la informaci\'on que nos facilita Spring:

\begin{itemize}
    \item Usuario - El usuario de Spring que utiliza la herramienta para el desarrollo abasteci\'endose de la informaci\'on de todos los tipos de manuales.
    \item Usuario activo - El usuario de Spring que se involucra en la comunidad registr\'andose en el foro, JIRA, StackOverFlow, etc, para buscar/preguntar una duda que le haya surgido.
    \item Usuario pro-activo - El usuario que adem\'as de preguntar, contesta a las dudas expuestas a trav\'es de los diferentes canales incluso subiendo parches que no aplicar\'a el mismo directamente. Este apartado merece una menci\'on especial ya que es el m\'as cuidado y m\'as importante dentro de la comunidad.
    \item Usuario JIRA - El usuario que propone mejoras a la herramienta mediante JIRA abriendo debates mediante tickets. \'estos tickets y cualquier otro son votados por la comunidad de JIRA para la implementaci\'on de los mismos o resoluci\'on de problemas entre la comunidad para poder establecer un orden de prioridad.
    \item Usuario Committer - El usuario que ayuda gestionando y resolviendo tickets que le asigna el grupo de desarrolladores encargados del proyecto. Aporta el c\'odigo fuente directamente habiendo aceptando el contrato para la subida de c\'odigo al repositorio de Spring siguiendo las directrices establecidas para la documentaci\'on y el formato.
\end{itemize}

No hay diferencia entre un usuario que est\'e trabajando en Spring y un usuario que est\'e indentificado con estos roles (igual el sueldo si) ya que est\'an dentro de una comunidad para sacar adelante el proyecto por lo que el intercambio de roles de un usuario depende de la actividad en la que \'este se implique en la comunidad.

Se ha de tener en cuenta que las aportaciones de c\'odigo fuente al proyecto relacionadas con las tareas del JIRA han de seguir un patr\'on de desarrollo y documentaci\'on establecido dentro de cada proyecto de Spring.

Una cosa importante, donde retomamos el rol de usuario pro-activo son las gu\'ias definidas para el uso de las herramientas de comunicaci\'on; el foro y el JIRA. Si las peticiones no siguen el formato adecuado quedar\'an desestimadas al instante. En resumen, no es necesario ser el miembro m\'as activo en esta comunidad para poder plantear una duda o aportar algo en el dise\~no de un proyecto de Spring, hace falta ser claro, conciso y esgrimir argumentos, es lo que se valora.

Por lo que cada rol tiene especificaciones definidas para su contexto que Spring se esfuerza en facilitar y promocionar por ello es una comunidad muy activa.

\begin{thebibliography}{9}

\bibitem{spring-inside}
  SpringSource Community,\\
  http://www.springsource.org/

\end{thebibliography}
\end{document}
