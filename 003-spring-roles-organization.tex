\documentclass[11pt]{scrartcl}

\title{\textbf{Legal Aspects\\
				Lesson 3}}
\subtitle{FLOSS Licensing}
\author{Ricardo Garc\'ia Fern\'andez}
\date{\today}

\begin{document}

\maketitle

\section{Exercise}

Analyze the \textbf{Spring} community to write a brief summary (1-2 pages) describing the main roles and organizational policies adopted in this community.\\

You must focus on the following points:

\begin{itemize}

	\item Different roles established in the community.
	\item Overall governance rules and guidelines.
	\item Promotion paths to access higher responsibilities in the community.
\end{itemize}

\section{Introduction to Spring}


Spring is an Open Source Framework under Apache License version 2.0\footnote{http://www.springsource.org/spring-framework}.

Spring es un Framework Open Source bajo Apache License version 2.0\footnote{http://www.springsource.org/spring-framework} creado a principios de la primera d\'ecada de los a\~nos 2000. Apareci\'o como proyecto publicado en Sourceforge en 2003 de la mano de Rod Johnson, su creador y que este mismo a\~no ha habandonado el proyecto para embarcarse en una nueva acometida\footnote{http://blog.springsource.org/2012/07/03/oh-the-places-youll-go/}.\\

Rod Johnson dise\~n\'o un framework para facilitar el desarrollo de proyectos utlizando Java aplicando las mejores pr\'acticas de dise\~no para los proyectos.\\

El lanzamiento de la versi\'on 1.0 a la comunidad fue muy bien aceptada debido a una caracter\'istica muy importante en este tipo de proyectos, la documentaci\'on. La cantidad y calidad de documentaci\'on que acompa\~naba al Framewrok ayud\'o a incrementar el uso en la comunidad, usuarios particulares y proyectos empresariales.
Esta caracter\'istica se sigue manteniendo y evolucionando d\'ia a d\'ia, mediante, tutoriales, foros, tickets, videos, charlas, certificados, en resumen, una muy buena comunicaci\'on con los usuarios del producto por lo que mantiene una gran comunidad asociada y potente.

\section{Primer vistazo}

Al acceder a la p\'agina del proyecto de Spring podemos ver un mensaje que invita a entrar y colaborar mediante dos vistosos botones: \emph{Get Started}\footnote{http://www.springsource.org/get-started} y \emph{Get Involved}\footnote{http://www.springsource.org/get-involved} por lo que por otra parte existen dos tipos de camino, el nuevo usuario y el usuario avanzado o habitual para interactuar con la comunidad.\\

Para comprender la estructura de la comunidad asociada al proyecto empezaremos analizando el camino del nuevo usuario.

\section{Get Started}

En este apartado de iniciaci\'on podemos encontrar todo tipo de informaci\'on referente al aprendizaje del proyecto Spring dividido en estos seis grupos:

\begin{itemize}

	\item Start a Tutorial - Tutoriales de uso.
    \item Grab a Code Sample - Ejemplos funcionales.
    \item Ask a Question (Forums) - Foro, activo e importante.
    \item Take a Class (Training) - "Universidad" de Spring para aprender a codificar.
    \item Read the Documentation - Important\'isimo apartado, no s\'olo leer si no, saber utilizar la documentaci\'on.
    \item Video Instruction - V\'ideos en los que muestra las herramientas y su uso en funcionamiento.
\end{itemize}

Como hemos remarcado anteriormente, Spring destac\'o y destaca por la calidad de su documentaci\'on y apoyo para la adopci\'on del Framework por parte de los desarrolladores.
Este proyecto nos invita a participar facilitando su comprensi\'on a un desarrollador Java desde distintos escalones de conocimiento.\\

Intenta abarcar diferentes m\'etodos de aprendizaje, lecturas, ejemplos, videos, preguntas, clases virtuales para que de esta manera no s\'olo exista una forma de comprensi\'on y los desarrolladores puedan elegir el camino que m\'as se adapte a ellos.\\

Se puede definir como una comunidad de aprendizaje del Framework Spring dentro de la misma comunidad Spring. Una subcomunidad en donde los usuario aprenden y pueden convertirse en profesores (utilizando la nomenclatura de la ense\~nanza) alrededor de un foro en el que orbitan los dem\'as servicios comunic\'andose entre ellos.

\section{Get Involved}

Aqu\'i es donde empieza el camino del usuario habitual o avanzado.\\

La presentaci\'on de esta p\'agina nos muestra un entusiasta comentario de bienvenida a la comunidad no importa el nivel de conocimiento que tengas de Spring ya que este es un grupo dedicado a aprender y a difundir los conocimientos.\\

Est\'a m\'as marcada la estructura escalonada por pasos con lo que se refirere a roles para un usuario inicial en esta etapa:

\begin{itemize}
    \item Join the conversarion - Anima a entrar en el ecosistema Spring para recibir informaci\'on del d\'ia a d\'ia.
    \item Help other users (And get help when you need it too) - Uso de los foros para ayudar o recibir ayuda a los dem\'as y tambi\'en mediante el uso de la red social StackOverFlow\footnote{http://stackoverflow.com/} con los tags spring y spring-mvc.
    \item Report issues - Informa de errores o mejoras a trav\'es del JIRA\footnote{http://jira.springsource.org/} de Spring.
    \item Track the latest features and test them out - Uso activo del JIRA para poder probar las nuevas caracter\'isticas o posibles errores ayudando a la comunidad a resolverlos.
    \item Contribute code - Nos emplaza al repositorio de c\'odigo de Spring en GitHub\footnote{http://github.com/SpringSource} para utilizar la última versi\'on del c\'odigo fuente publicado para que a trav\'es de los tickets definidos en Github y su relaci\'on con el JIRA puedas aportar una soluci\'on o mejora a trav\'es de tu c\'odigo fuente y si \'este soluciona o satisface los requisitos, entrar\'as a formar parte del grupo de usuarios que aportan c\'odigo a srping framewor mediante la firma de un contrato de contribuyente \footnote{https://support.springsource.com/spring\_committer\_signup} al proyecto.
    \item Attend (or give) a talk at a local user group - Anima al desarrollador a ir a charlas sobre Spring o a ser el ponente de la misma charla en tus grupos cercanos de desarrolladores destacando que Spring aprecia mucho esta tarea, la ayudar a promover el buen uso de su Framework ofreciendo directamente su ayuda.
    \item Attend a spring-related conference - Como último punto nos da informaci\'on sobre la conferencia anual de Spring donde se informa y se debate sobre las mejoras y/o novedades de Spring uni\'endote a ella para estar m\'as en contacto con la comunidad.
\end{itemize}

\begin{thebibliography}{9}

\bibitem{spring-inside}
  SpringSource Community,\\
  http://www.springsource.org/

\end{thebibliography}
\end{document}
