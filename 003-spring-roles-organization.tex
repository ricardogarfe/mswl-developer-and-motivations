\documentclass[11pt]{scrartcl}

\title{\textbf{Legal Aspects\\
				Lesson 3}}
\subtitle{FLOSS Licensing}
\author{Ricardo Garc\'ia Fern\'andez}
\date{\today}

\begin{document}

\maketitle

\section{Exercise}

Analyze the \textbf{Spring} community to write a brief summary (1-2 pages) describing the main roles and organizational policies adopted in this community.\\

You must focus on the following points:

\begin{itemize}

	\item Different roles established in the community.
	\item Overall governance rules and guidelines.
	\item Promotion paths to access higher responsibilities in the community.
\end{itemize}

\section{Introduction to Spring}

Spring is an Open Source Framework under Apache License version 2.0\footnote{http://www.springsource.org/spring-framework} 

creado a principios de la primera década de los años 2000. Apareció como proyecto publicado en Sourceforge en 2003 de la mano de Rod Johnson, su creador y que este año ha habandonado el proyecto para embarcarse en una nueva acometida\footnote{http://blog.springsource.org/2012/07/03/oh-the-places-youll-go/}.

Rod Johnson diserñó un framework para facilitar el desarrollo de proyectos utlizando Java aplicando las mejores prácticas de diseño para los proyectos. 

El lanzamiento de la versión 1.0 a la comunidad fue muy bien aceptada debido a una característica muy importante en este tipo de proyectos, la documentación. La cantidad y calidad de documentación que acompañaba al Framewrok ayudó a incrementar el uso en la comunidad, usuarios particulares y proyectos empresariales.
Esta característica se sigue manteniendo y evolucionando día a día, mediante, tutoriales, foros, tickets, videos, charlas, certificados, en resumen, una muy buena comunicación con los usuarios del producto por lo que mantiene una gran comunidad asociada y potente.

\section{Primer vistazo}

Al acceder a la página del proyecto de Spring podemos ver un mensaje que invita a entrar y colaborar mediante dos vistosos botones: \emph{Get Started}\footnote{http://www.springsource.org/get-started} y \emph{Get Involved}\footnote{http://www.springsource.org/get-involved} por lo que por otra parte existen dos tipos de camino, el nuevo usuario y el usuario avanzado o habitual para interactuar con la comunidad.

Para comprender la estructura de la comunidad asociada al proyecto empezaremos analizando el camino del nuevo usuario.

\section{Get Started}

En este apartado de iniciación podemos encontrar todo tipo de información referente al aprendizaje del proyecto Spring dividido en estos seis grupos:

    Start a Tutorial - Tutoriales de uso.
    Grab a Code Sample - Ejemplos funcionales.
    Ask a Question (Forums) - Foro, activo e importante.
    Take a Class (Training) - "Universidad" de Spring para aprender a codificar.
    Read the Documentation - Importantísimo apartado, no sólo leer si no, saber utilizar la documentación.
    Video Instruction - Videos en los que muestra las herramientas y su uso en funcionamiento.

Como hemos remarcado anteriormente, Spring destacó y destaca por la calidad de su documentación y apoyo para la adopción del Framework por parte de los desarrolladores.
Este proyecto nos invita a participar facilitando su comprensión a un desarrollador Java desde distintos escalones de conocimiento.

Intenta abarcar diferentes métodos de aprendizaje, lecturas, ejemplos, videos, preguntas, clases virtuales para que de esta manera no sólo exista una forma de comprensión y los desarrolladores puedan elegir el camino que más se adapte a ellos.

Se puede definir como una comunidad de aprendizaje del Framework Spring dentro de la misma comunidad Spring. Una subcomunidad en donde los usuario aprenden y pueden convertirse en profesores (utilizando la nomenclatura de la enseñanza) alrededor de un foro en el que orbitan los demás servicios comunicándose entre ellos.

\section{Get Involved}

Aquí es donde empieza el camino del usuario habitual o avanzado. 

La presentación de esta página nos muestra un entusiasta comentario de bienvenida a la comunidad no importa el nivel de conocimiento que tengas de Spring ya que este es un grupo dedicado a aprender y a difundir los conocimientos.

Está más marcada la estructura escalonada por pasos con lo que se refirere a roles para un usuario inicial en esta etapa:

    Join the conversarion - Anima a entrar en el ecosistema Spring para recibir información del día a día.
    Help other users (And get help when you need it too) - Uso de los foros para ayudar o recibir ayuda a los demás y también mediante el uso de la red social StackOverFlow\footnote{http://stackoverflow.com/} con los tags spring y spring-mvc.
    Report issues - Informa de errores o mejoras a través del JIRA\footnote{http://jira.springsource.org/} de Spring.
    Track tehe latest features and test them out - Uso activo del JIRA para poder probar las nuevas características o posibles errores ayudando a la comunidad a resolverlos.
    Contribute code - Nos emplaza al repositorio de código de Spring en GitHub\footnote{http://github.com/SpringSource} para utilizar la última versión del código fuente publicado para que a través de los tickets definidos en Github y su relación con el JIRA puedas aportar una solución o mejora a través de tu código fuente y si éste soluciona o satisface los requisitos, entrarás a formar parte del grupo de usuarios que aportan código a srping framewor mediante la firma de un contrato de contribuyente \footnote{https://support.springsource.com/spring\_committer\_signup} al proyecto.
    Attend (or give) a talk at a local user group - Anima al desarrollador a ir a charlas sobre Spring o a ser el ponente de la misma charla en tus grupos cercanos de desarrolladores destacando que Spring aprecia mucho esta tarea, la ayudar a promover el buen uso de su Framework ofreciendo directamente su ayuda.
    attend a spring-related conference - Como último punto nos da información sobre la conferencia anual de Spring donde se informa y se debate sobre las mejoras y/o novedades de Spring uniéndote a ella para estar más en contacto con la comunidad.

\begin{thebibliography}{9}

\bibitem{spring-inside}
  SpringSource Community,\\
  http://www.springsource.org/

\end{thebibliography}
\end{document}
