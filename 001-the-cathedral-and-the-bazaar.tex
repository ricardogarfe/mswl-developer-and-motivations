\documentclass[11pt]{scrartcl}
\usepackage{atbegshi,picture}
\usepackage{lipsum}

\AtBeginShipout{\AtBeginShipoutUpperLeft{%
  \put(\dimexpr\paperwidth-1cm\relax,-1.5cm){\makebox[0pt][r]{Master on Free Software 2012/2013}}%
}}

\title{\textbf{Developers and motivations}}
\subtitle{Questions about "The Cathedral and the Bazaar"}

\author{Felipe Ortega\\
		Ricardo Grac\'ia Fern\'andez}
\date{\today}
\begin{document}

\maketitle

Revise the content of Eric S. Raymond's essay to answer the following points:
	\begin{enumerate}
		\item In your opinion, which are the 2 most important lessons that we can find in the essay? Provide links and examples to backup your reasoning (1 to 10).\\
		
			\textbf{Answer:}\\
			
			First, in my opinion I think that the first 'rule' is most important:
			\begin{itemize}
				\item 1. Every good work of software starts by scratching a developer's personal itch.
			\end{itemize}
			
			\indent Is most valuable, not only for \textbf{FLOSS} developers but also applies to any field because everyone who wants to do something, has to want to do this thing. This could sound weird and have nonsense because it's obvious but you have to be willing to do something to start doing something that interests you, this is the most important not to fall into disuse.
			
			\indent As shown in \emph{Git History}\footnote{http://git-scm.com/book/en/Getting-Started-A-Short-History-of-Git}, in 2005 Linux kernel developers needed a new \emph{DVCS}\footnote{http://en.wikipedia.org/wiki/Distributed\_revision\_control} because of his broke up relation with \emph{BitKeeper}\footnote{http://www.bitkeeper.com/} and then started to develop its own DVCS called Git. Nowadays its the more extended and adopted DVCS worldwide.\\
			
			The second I choose is the eighth one:
			\begin{itemize}
				\item 8. Given a large enough beta-tester and co-developer base, almost every problem will be characterised quickly and the fix obvious to someone.
			\end{itemize}
			
			\indent This rule is very important for a \textbf{FLOSS} developer in every project, you need more eyes not only yours to see how is going your development, users, testers, coders, in summary you need help and have to be capable to admit it and take advantage to make things better and easier.
			
			\indent A sample of spread code could be \emph{Android SDK}\footnote{http://source.android.com/source/downloading.html}. I know, Google only release source code when he wants, but this is not the question, is how code release help your development increases capabilities, gets strong and it's used for more and more people and therefore your beta-tester 'database' increases everyday.\\
			
		\item Now, find out the 2 most controversial lessons. Again, try to include links and examples to support your comments (1 to 19).

			\textbf{Answer:}\\
			
			The most controversial poitn for me is:
			
			\begin{itemize}
			
				\item 6. Treating your users as co-developers is your least-hassle route to rapid code improvement and effective debugging.
			\end{itemize}
			
			Because Eric somehow assumes that its users or users in general are capable to do all this stuff magically. I think if you want to retrieve good co-developers, beta testers to create a effective debugging you have to train them with necessary knowledge. 
			
			If you only get every \emph{comment} from your co-stuff you have to invest more time to check all the information arrives to you. You have to make work easier for \emph{FLOSS} developers. In this case we can take the snapshot from: \\
			
			In the other hand we have next point or rule:
			
			\begin{itemize}
			
				\item 5. When you lose interest in a program, your last duty to it is to hand it off to a competent successor.
			\end{itemize}
			
			In this case I think if you have to look for a successor, your project is dead. Let me explain, if you, somehow loose the interest in your project that you started to build with all your effort and desire to create, nobody can treat your project in the same way, only in the case that this person wants something to deal with that is important for her/him.
			
			The main sample opposite to the rule is \emph{Popmail} that was abandoned until somebody wanted to do something for its interest, Eric Raymon and become \emph{Fetchmail}\footnote{http://www.fetchmail.info/}.
			
	\end{enumerate}

\begin{thebibliography}{9}

	\bibitem{lamport94}
		The Cathedral and the Bazaar,\\
		\emph{Eric Steven Raymond}.\\
		http://www.catb.org/esr/writings/homesteading/cathedral-bazaar/index.html
	
	\bibitem{lamport94}
		A Second Look at The Cathedral and the Bazaar,\\
		\emph{Nikolai Bezroukov}.\\
		http://firstmonday.org/htbin/cgiwrap/bin/ojs/index.php/fm/article/view/708/618	
\end{thebibliography}

\end{document}